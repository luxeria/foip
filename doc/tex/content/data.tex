\section{Daten laden und speichern}
Einstieg in jegliche Bildverarbeitung ist das Einlesen der Bilddaten und
auch deren Speicherung. Im einfachsten Fall kann der Prozess in drei Teile
gegliedert werden:

\begin{enumerate}
	\item Originaldatei einlesen
	\item Manipulation
	\item Manipulierte Daten ablegen
\end{enumerate}

\subsection{Graustufenbilder}
Diese drei elementaren Schritte sind hier mit den beiden Funktionen 
\lstinline{imread()} und \lstinline{imwrite()} des image package ein
einem Pseudobeispiel dargestellt:

\begin{lstlisting}
I = imread("input_image.png");		% read image
J = my_image_processing(I);		% manipulate image
imwrite(J, "output_image.png");		% write image
\end{lstlisting}

Durch die Verwendung von \lstinline{imread()} auf Graustufenbilder wird
die Grafik in eine entsprechende Matrix gespeichert mit den Pixelwerten.
Diese kann per \lstinline{imshow()} betrachtet werden.

\subsection{RGB Dateien}
Beim Einlesen von Graustufenbildern sind die Pixelwerte direkt in einer
2D Matrix ableget. Beim Einlesen von RGB-Bildern wird die Grafik in eine
mehrschichtige 2D Matrix ableget, wobei jede Matrix die Pixelwerte für
eine Farbe enthält. Die einzelnen Matrizen können wiederum als
Graustufenbilder behandelt werden. Im Folgenden wird das Zerlegen eines
RGB-Bildes in die einzelnen Farblayer aufgezeigt.

\begin{lstlisting}
Image = imread("input_rgb_image.png");	% read the RGB image

Red   = Image(:,:,1);			% extract red layer
Green = Image(:,:,2);			% extract green layer
Blue  = Image(:,:,3);			% extract blue layer
\end{lstlisting}
