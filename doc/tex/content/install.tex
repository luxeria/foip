\section{Software-Installation}
Für die in diesem Dokument behandelten Code-Beispiele wird GNU Octave
verwendet und das image Softwarepaket. Die Installation von GNU Octave
erfolgt bei Unixoiden Systemen über den Paketmanager und kann direkt
per Kommandozeile erfolgen. Hier das Beispiel für den Paketmanager
pacman von Archlinux und aptitude von Ubuntu (Debian):

\begin{lstlisting}
sudo pacman -S octave
\end{lstlisting}

\begin{lstlisting}
sudo apt-get install octave
\end{lstlisting}

\subsection{Forge}
Die Instalation von Erweiterungen von Octave kann direkt aus der
Kommandozeile von Octave erfolgen. Hierzu kann entweder das gewünschte
Softwarepaket manuell lokal heruntergeladen und danach installiert werden
oder direkt aus der Datenbank bezogen und installiert werden durch
Octave.

\begin{lstlisting}
pkg install -forge image
\end{lstlisting}
